\section*{\zihao{-3}\heiti 泰坦尼克号幸存者数据分析及基于神经网络填充缺失值的预测模型}
\section*{\zihao{3}\heiti 摘\ \ \ \ \ \ \ \ 要}

泰坦尼克号是英国白星航运公司20世纪10年代建设的一搜豪华游艇。其排水量达到了惊人的4.6万吨,是当时世界上体积最大,最豪华的客运轮船。但正是这艘号称“永不沉没”的泰坦尼克号,在1912年从英国驶往美国时,在大西洋与冰山相撞并沉没!

根据英国贸易委员会公布的数据显示,在灾难发生时,泰坦尼克号共搭载2224人,其中710人生还,1514人不幸罹难,其中乘客约有1317人,共498人幸存;男性船员约有885人,共192人幸存;女性船员23人,共20人幸存。

本文通过数据分析与机器学习算法,对泰坦尼克号数据进行数据预处理,数据可视化,特征工程,模型调参,模型优化,有效提取了泰坦尼克号数据中的信息,并对幸存者建立了预测模型,模型准确率高达85.86\%。

\textbf{第一步}:首先进行导包,读取数据,然后查看数据基本信息,并对每个属性的数据进行剖析。\textbf{第二步}:对多个属性之间的关系进行分析,探索与可视化,得出幸存率主要与性别,年龄,船舱有关的结论。\textbf{第三步}:首先对无用信息、非数值型数据、缺失数据进行处理,然后训练神经网络模型并对Age数据进行预测填充。\textbf{第四步}:进行模型的训练与验证,模型调参与模型融合。
\par
 
 
 \par
%空一行

\noindent \zihao{-4}{\heiti 关键词:}\textbf{泰坦尼克号};\textbf{特征工程};\textbf{参数优化};\textbf{数据分析};\textbf{神经网络}

\clearpage


%\section*{\zihao{-3}\bfseries Research on Microstructure and Mechanical Properties of Mg/Al Composite Sheet during Rolling Process}

%\hfill Author: xxxx

%\hfill Tutor: xxxx

%\section*{\zihao{3}\bfseries Abstract}

%As a green material in the 21st century, magnesium alloys have the advantages of low density, high specific strength, excellent thermal conductivity, good damping, shock absorption and impact resistance. However, the low yield strength, high notch sensitivity and poor corrosion resistance of magnesium alloys greatly limit the large-scale industrial application of them. Al/Mg/Al composite sheets have high specific strength and corrosion resistance of magnesium alloys with the advantages of good plasticity, toughness and corrosion resistance of aluminum and its alloys. It has broad application prospects in automotive, aerospace and other fields. In this paper, Al/Mg/Al composite sheets were prepared by composite rolling based on AZ31B magnesium alloy and 1060 pure aluminium sheets. The effects of process parameters on the microstructure and mechanical properties of the composite sheets were revealed.
%	\\
	%空一行
%	\\
%	\zihao{-4}{\bfseries Key words:\ }AZ31B magnesium alloy; Composite rolling; Microstructure; Mechanical property; Intermetallic compounds

%\clearpage




