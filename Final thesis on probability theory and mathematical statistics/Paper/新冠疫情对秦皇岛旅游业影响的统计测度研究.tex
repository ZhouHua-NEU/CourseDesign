%!TEX program = xelatex
\documentclass[11pt,punct,twoside]{ctexart}
\usepackage{titling}
\usepackage[a4paper, left=3cm, right=2cm, top=3cm, bottom=2.5cm]{geometry}
\usepackage{fancyhdr, graphicx, xpatch, layout}
\usepackage{booktabs}
\usepackage{amsmath}
\usepackage{pdfpages}
\usepackage{float}
%\usepackage{cite}
\usepackage{enumitem}
\usepackage{titlesec}
\usepackage{titletoc}
\usepackage{tabularx}
\usepackage{chngcntr}
\usepackage{listings}
\usepackage{cleveref}
\usepackage{subfigure}
\usepackage[titles,subfigure]{tocloft}
\usepackage[labelsep=space]{caption}
\usepackage[numbers,super,square,comma,sort,compress]{natbib}
\usepackage{lmodern}
\usepackage{indentfirst}
\usepackage{xtab,booktabs}
\usepackage[version=4]{mhchem}
\usepackage{array}
\usepackage{fontspec}
\usepackage{caption}
\usepackage{xeCJK}
\usepackage{color}
\usepackage{longtable}
\usepackage{ctex}
%\usepackage{longtabu}
\usepackage{tabu}
\usepackage{makecell}
\usepackage{amssymb}
%\usepackage{showframe}

% 微分算子
\newcommand*{\dif}{\mathop{}\!\mathrm{d}}

% 参考文献行间距
\setlength{\bibsep}{-1pt}
%设置字体
\setmainfont{Times New Roman} % set all eng font
\songti%正文宋体
\zihao{-4}%正文小四号
\ctexset{section={format+={\mdseries\zihao{3}\heiti},aftername = \hspace{0.5em}}}%章标题 三号黑体
\ctexset{subsection={format+={\mdseries\zihao{-4}\heiti},aftername = \hspace{0.5em}}}%节标题 小四号黑体
\ctexset{subsubsection={format+={\mdseries\zihao{-4}\heiti},aftername = \hspace{0.5em}}}%级标题 小四号黑体  

%宋体伪粗体
\setCJKfamilyfont{zhsong}[AutoFakeBold = {2.17}]{SimSun}
\renewcommand*{\songti}{\CJKfamily{zhsong}}
%设置间距
\renewcommand{\baselinestretch}{1.62}
%\linespread{1.5}
\titlespacing{\section}{0bp}{0.5em}{0.5em}
\titlespacing{\subsection}{0bp}{0.5em}{0.5em}
\titlespacing{\subsubsection}{0bp}{0.5em}{0.5em}
%\setlength{\parskip}{0.5\baselineskip}
% 图表公式每章重新编号
\counterwithin{table}{section}
\counterwithin{figure}{section}
\counterwithin{equation}{section}

%表标题与内容间距
\setlength{\abovecaptionskip}{0.1cm}
%列表样式
\setlist[enumerate,1]{label=(\arabic*), wide, labelsep=0.5em, itemsep=0ex, topsep=0pt, partopsep=0pt, parsep=0pt}
\setlist[enumerate,2]{label=(\alph*) , wide, labelsep=0.5em, itemsep=0ex, topsep=0pt, partopsep=0pt, parsep=0pt,itemindent=5em}

%正文代码块lstlisting样式
\def\inline{\lstinline[basicstyle=\ttfamily,breaklines=true]}
\lstset{
	basicstyle=\ttfamily\zihao{5},
	xleftmargin=2em,
	%xrightmargin=2em,
    breaklines=true,
    frame={tb},
    belowcaptionskip=0.6em,
    framerule=1.5pt
}

%引用样式
\renewcommand{\lstlistingname}{代码清单}
\crefname{listing}{代码清单}{代码清单}
\crefname{table}{表}{表}
\Crefname{table}{表}{表}
\crefname{figure}{图}{图}
\Crefname{figure}{图}{图}

%自定义命令
\newcommand{\scite}[1]{\textsuperscript{\cite{#1}}} %cite样式
\DeclareCaptionFormat{myformat}{\zihao{5}\selectfont#1#2#3}
\captionsetup{format=myformat}
\captionsetup{labelsep=quad}%图表号与图表名之间空两格

%------------------文档开始------------------
\begin{document}
\pagestyle{empty}
\begin{titlepage}
	\begin{figure}[t]
		\centering
		\includegraphics[width=0.92\textwidth]{images/cover.png}
	\vspace{-2.5em}
	\end{figure}
	
	\begin{center}
		\quad \\
		\quad \\
		\heiti \zihao{1} 《医疗数据处理实践》课程设计报告
		\vskip 0.5em
		\heiti \zihao{2} 泰坦尼克号幸存者数据分析及基于神经网络填充缺失值的预测模型
	\end{center}
	\vskip 1em
	
	\makeatletter
	
	\newcommand\dlmu[2][8em]{\underline{\hb@xt@ #1{\hss#2\hss}}}
	\makeatother	
	\begin{center}
		\zihao{-3}
		\heiti
		\renewcommand\arraystretch{0.72} 
		\setlength\extrarowheight{5mm} 
		\begin{tabular}[b]{b{-0.5em}b{4em}b{0.25em}b{16em}<{\centering}}
			%\cline{4-4}
			&\makebox[4em][s]{学\ \ \ \ \ \ \ \ 院}  &\ 	& 数\ 学\ 与\ 统\ 计\ 学\ 院 \\
			\cline{4-4}
			&\makebox[4em][s]{专\ \ \ \ \ \ \ \ 业}	&\ 	&数\ 据\ 科\ 学\ 与\ 大\ 数\ 据\ 技\ 术     \\
			\cline{4-4}
			&\makebox[4em][s]{班级序号}	&\ 	&\bf{200221}   \\
			\cline{4-4}
			&\makebox[4em][s]{学\ \ \ \ \ \ \ \ 号}	&\ 	&\bf{202015140}   \\
			\cline{4-4}
			&\makebox[4em][s]{姓\ \ \ \ \ \ \ \ 名}	&\ 	&周\ 华   \\ 
			\cline{4-4}
			&\makebox[4em][s]{指导教师}	&\ 	&王\  子\ 健\ 、\ 张\  建\ 波\ \\
			\cline{4-4}
			&\makebox[4em][s]{开始日期}	&\ 	&2022\  年\ 5\ 月\  28\ 日\ \\
			\cline{4-4}
			&\makebox[4em][s]{结束日期}	&\ 	&2022\  年\ 7\ 月\  6\ 日\ \\
			\cline{4-4}
		\end{tabular}
	\end{center}



		\vskip 1em
		\centering
\end{titlepage}
\clearpage
%\include{sections/statement}

%页眉页脚
\pagestyle{fancy}
%\setlength{\voffset}{3pt}
\lhead{}
\chead{\zihao{4}\heiti 概率论与数理统计(含随机过程)结课论文~~~~~~~~~~~~~~~~~~~~~~~~~~~~~~~~~~~~~~~~~~~~~~~~ \includegraphics[scale=0.15]{images/logo.png}}%\bfseries 不用加粗


\rhead{}
\lfoot{}
\cfoot{\zihao{5}\thepage}
\rfoot{}

\setlength{\voffset}{-10mm}                        
\setlength{\topmargin}{0mm}
\setlength{\headheight}{6mm}
\setlength{\headsep}{9mm}
\setlength{\footskip}{7.5mm}
\pagenumbering{Roman} %目录页码为罗马数字
\setcounter{page}{1}%页码重新计数

%正文代码块章节编号
\counterwithin{lstlisting}{section}

\section*{\zihao{-3}\heiti 基于Bagging模型的回归优化模型}
\section*{\zihao{3}\heiti 摘\ \ \ \ \ \ \ \ 要}

回归问题在生活和工业中非常重要。本文将会以一个简单的数学问题出发,采用不同的回归方法进行回归并对不同的回归方法进行比较。从而探究几种常见的回归方法。

本文第一部分介绍算法的基本原理,包括最小二乘法原理,机器学习回归原理,\textbf{KNN}回归原理,神经网络回归原理,同时介绍了B\textbf{agging}集成算法的原理。

本文第二部分开始使用四种回归方法对2020年明信片的价格分别进行回归并预测。然后使用\textbf{Bagging}算法求出3种回归算法的平均值。

本文第三部分首先将数据划分为两组。用四种回归算法和\textbf{Bagging}集成算法进行回归,然后计算其均方差。结果显示,使用神经网络回归的效果最佳。其均方差只有$\textbf{32.20}$,因此,预测$2020$年明信片的价格为$\textbf{34.00}$美分。

%空一行

\zihao{-4}{\heiti 关键词:\textbf{最小二乘法} ~~ \textbf{KNN回归} ~~ \textbf{高阶多项式回归} ~~ \textbf{剃度下降法} ~~ \textbf{Bagging模型}} 

\clearpage






\include{sections/toc}
\pagenumbering{arabic} %正文页码为阿拉伯数字
\section{绪论}
回归,指研究一组随机变量$(y_1 ,y_2 ,\dots,y_i)$和另一组$(x_1,x_2,\dots,x_k)$变量之间关系的统计分析方法,又称多重回归分析。通常$y_1,y_2,\dots,y_i$是因变量,$x_1,x_2,\dots,x_k$是自变量。常见的回归算法包括最小二乘法回归,机器学习回归,KNN回归,神经网络回归等。

在回归问题中,对于某些离群值或者异常值,单个回归算法会产生较大误差。集成算法是一类构建多个学习器,通过一定策略结合来提高准确率,降低误差的模型。常见的集成算法分为Bagging、Boosting、Stacking三大类。本文主要使用了Bagging算法。

\subsection{回归算法}
\subsubsection{最小二乘法回归}
给定$n$个属性描述的自变量$\textbf{x}=(x_1,x_2,$\dots$,x_n)$,其中$x_i$为\textbf{x}在第$i$个元素上的取值。线性回归试图学得一个通过该属性的线性组合来进行预测的函数,即

\begin{equation}
f(\textbf{x})=w_1x_1+w_2x_2+\dots+w_nx_n+b, \label{1}
\end{equation}


一般用向量形式写成
\begin{equation}
f(\textbf{x})=\textbf{w}^T\textbf{x}+b,\label{2}
\end{equation}

其中$\textbf{w}=(x_1,x_2,$\dots$,x_n)$,$\textbf{w}$和$b$确定后,模型也得以确定。

给定数据集$D={((\textbf{x}_1,y_1),(\textbf{x}_2,y_2),\dots,(\textbf{x}_n,y_n)}$,其中$\textbf{x}_i=(x_{i1},x_{i2},\dots,x_{im}),y_i \in \mathbb{R}$.线性模型试图学得

\begin{equation}
f(\textbf{x}_i)=\textbf{w}^T\textbf{x}_i+b,~~~~~~~~s.t. ~~~~~~~~f(\textbf{x}_i)\approx y_i .\label{3}
\end{equation}
为了方便,我们令
 $$\mathop{w}\limits^{\^{}}=(\textbf{w},b) ,$$
 
 $$\textbf{x}=\begin{pmatrix}
 x_{11}&  x_{12}&  \dots&  x_{1d}& 1\\
 x_{21}&  x_{22}&  \dots&  x_{2d}& 1\\
 \vdots&  \vdots& \ddots  &  \vdots& \vdots\\
 x_{m1}&  x_{m2}&  \dots&  x_{md}& 1
 \end{pmatrix}=\begin{pmatrix}
 x_{1}^T& 1\\
 x_{2}^T& 1\\
 \vdots& \vdots\\
 x_{m}^T& 1
 \end{pmatrix},$$
 
 $$\textbf{y}=(y_1,y_2,\dots,y_m),$$
 其中,$\textbf{x}$是一个$m \times (d+1)$ 的矩阵,每行前$d$个元素对应集合$D$中的$d$个属性值。
 
 为了实现 \ref{3} ,我们使用均方误差来衡量$f(x_i)$与$y_i$之间的差别,以便确定最合适的$\textbf{w}$和$b$:
 \begin{equation}
E(f,D)=\frac{1}{m}\sum_{i=1}^m(f(x_i)-y_i)^2.\label{4}
 \end{equation}

 线性回归的任务转化为
 
 \begin{equation}
 \mathop{w}\limits^{\^{}}^*=arg_{\mathop{w}\limits^{\^{}}}min(\textbf{y}-\textbf{x}\mathop{w}\limits^{\^{}})^T(\textbf{y-x}\mathop{w}\limits^{\^{}}).
 \label{5}
 \end{equation}
 
令$E_{\mathop{w}\limits^{\^{}}}=(\textbf{y}-\textbf{x}\mathop{w}\limits^{\^{}})^T(\textbf{y-x}\mathop{w}\limits^{\^{}})$,对$\mathop{w}\limits^{\^{}}$求导得到
\begin{equation}
\frac{\partial E_{\mathop{w}\limits^{\^{}}}}{\partial \mathop{w}\limits^{\^{}}}=2\textbf{x}^T(\textbf{x}\mathop{w}\limits^{\^{}}-\textbf{y}).
\label{6}
\end{equation}

令\ref{6}等于零可以求得$\mathop{w}\limits^{\^{}}$最优解的闭式解。特殊地,当集合域D中只有一个属性时,可以求得
\begin{equation}
w=\frac{\sum\limits_{i=1}^my_i(x_i-\mathop{x}\limits^{-})}{\sum\limits_{i=1}^mx_i^2-\frac{1}{m}(\sum\limits_{i=1}^mx_i)^2},
\label{7}
\end{equation}

\begin{equation}
b=\frac{1}{m}\sum\limits_{i=1}^m(y_i-wx_i),
\label{8}
\end{equation}
其中$\mathop{x}\limits^{-}=\frac{1}{m}\sum\limits_{i=1}^mx_i$为$x$的均值.

现在对一个给定的数据点集用最小二乘法准则拟合$y=Ax^n$形式的曲线,n为固定数,研究模型$f(x)=ax^n$的最小二乘估计,应用该准则要求极小化$$s=\sum_{i=1}^m[y_i-f(x_i)]^2=\sum_{i=1}^m[y_i-ax_i^n]^2$$.最优化的必要条件是$\frac{dS}{da}=-2\sum_{i=1}^mx_i^n[y_i-ax_i^n]=0$,从方程可以解出a,得$$a=\frac{\sum x_i^ny_i}{\sum x_i^{2n}}$$
 






\subsubsection{机器学习回归}


在1.1.1中为了实现\ref{3},我们用均方误差\ref{4}来衡量拟合的曲线与实际数据的误差.当我们使\ref{4}取得最小值时,也就确定了最优解.神经网络使用梯度下降算法确定最优解.为了求出\ref{5},神经网络模型的计算步骤如下:



\clearpage


\begin{algorithm}[h]
	\caption{机器学习回归问题}
	\label{alg:4}
	\begin{algorithmic}[1]
		\STATE 随机初始化一组值$w_0$\
		\STATE 求解$\frac{\partial E_{\mathop{w}\limits^{\^{}}}}{\partial \mathop{w}\limits^{\^{}}}=2\textbf{x}^T(\textbf{x}\mathop{w}\limits^{\^{}}-\textbf{y})$\
		\STATE $w_1=w_0-\alpha \frac{\partial E_{\mathop{w}\limits^{\^{}}}}{\partial \mathop{w}\limits^{\^{}}}$,其中$\alpha$为常数,也称学习速率.
		\STATE $w_0=w_1$,重复3,直到$w_1<10^{-6}$
		\STATE 输出结果$w_1$
	\end{algorithmic}
\end{algorithm}





其中,特别地,当集合域D中只有一个属性时:
\begin{algorithm}[h]
	\caption{只有一个属性时机器学习解决回归问题}
	\label{alg:4}
	\begin{algorithmic}[1]
		\STATE 随机初始化一组值$w_0,b_0$\
		\STATE 求解$\frac{\partial E_{\mathop{w}\limits^{\^{}}}}{\partial \mathop{w}\limits^{\^{}}}$,$\frac{\partial E_{\mathop{w}\limits^{\^{}}}}{\partial \mathop{b}}$
		
		
		
		\STATE $w_1=w_0-\alpha \frac{\partial E_{\mathop{w}\limits^{\^{}}}}{\partial \mathop{w}\limits^{\^{}}}$,$b_1=b_0-\alpha \frac{\partial E_{\mathop{w}\limits^{\^{}}}}{\partial \mathop{b}\limits^{\^{}}}$其中$\alpha$为常数,也称学习速率.
		\STATE $w_0=w_1$,$b_0=b_1$,重复3,直到$w_1<10^{-6}$
		\STATE 输出结果$w_1,w_0$
	\end{algorithmic}
\end{algorithm}

机器学习能够解决的不仅只有线性回归问题,当知道目标函数后,即可使用剃度下降法求得目标函数的极值,从而求解各种回归问题。


\subsubsection{KNN回归}
KNN算法不仅可以用来聚类,还可以用来回归。KNN是一个简单的算法。简单的说就是取x距离最近的K个点,求它们的平均值作为预测值。
如图\ref{1}是一个K=3时用KNN回归的一个例子,KNN算法选取了最近的3个样本并取它们的平均值作为预测值。

\begin{figure}[h]
	\centering
	\includegraphics[scale=0.5,angle=0]{images/1.png}
	\caption{K=3时KNN回归的一个例子}
	\label{1}
\end{figure}


使用KNN回归的算法流程图如下:

\begin{document}
	% 例1
	\begin{algorithm}[h]
		\caption{KNN回归}
		\label{alg:Framwork}
		\begin{algorithmic}[1] %这个1 表示每一行都显示数字
			\REQUIRE ~~\\ %算法的输入参数:Input
			N个训练样本的集合  $A[n]$;\\
			人为设定的近邻个数$k$;\\
			给定的需要的预测的变量$x$:\\
			\ENSURE ~~\\ %算法的输出:Output
			预测的值$y$;
			\STATE 选择$A[1]$至$A[k]$作为$x$的初始近邻;
			\STATE 计算初始近邻与测试样本x间的欧氏距离$d(x, A[i]), i=1,2,...k$;
			\STATE 按$d(x, A[i])$从小到大排序;
			\STATE 计算最远样本与$x$间的距离$D$,即$max \{ d(x, A[j]) | j=1,2...k \} $;
			\STATE 
			
			 $for(i=k+1; i<n+1; i++)$ \{ 
			 
			 ~~~~~~~~~~~~~计算$A[i]$与$x$间的距离$d(x, A[i])$;
			  
			 ~~~~~~~~~~~~~$if (d(x, A[i]) < D )$
			 
			  ~~~~~~~~~~~~~~~~~~~~~~~~~~\{ 
			  
			  ~~~~~~~~~~~~~~~~~~~~~~~~~~用$A[i]$代替最远样本;  
			  
			  ~~~~~~~~~~~~~~~~~~~~~~~~~~ \}
			 
			 ~~~~~~~~~~~~~按照d(x, A[i])从小到大排序;
			 
			 ~~~~~~~~~~~~~计算最远样本与x间的距离D,即$max\{d(x, A[j]) | j=1,...i\}$;
			 
			 ~~~~~~~~~~~~~~~~~~~~~~~~~~~~~~~~~~~~~~~~~~~~~~~~~~~~\}
			
			\STATE 计算$y=\frac{1}{k} \sum_{i=1}^kd_i$
			\RETURN $y$; %算法的返回值
		\end{algorithmic}
	\end{algorithm}

\subsubsection{神经网络回归}
神经网络(Artificial Neural Networks,简写为ANNs)也简称为神经网络(NNs)或称作连接模型(Connection Model),它是一种模仿动物神经网络行为特征,进行分布式并行信息处理的算法数学模型。

神经网络在理论上能够拟合任意函数,包括非线性函数,是当下最流行的数学建模方法之一。

1959年两个生物科学家发现青蛙的神经元接受多个输入,输入包括青蛙的多个器官的输入,只有单输入的和到达一个阈值,才会有输出(青蛙接受的刺激比较大时才会有反应。

\begin{figure}[h]
	\centering
	\includegraphics[scale=0.5,angle=0]{images/2.png}
	\caption{科学家研究青蛙的神经元原理}
	\label{2}
\end{figure}


于是计算机科学家仿照生物神经元的原理和结构,提出了感知器:
\begin{figure}[h]
	\centering
	\includegraphics[scale=0.6,angle=0]{images/3.png}
	\caption{感知器}
	\label{3}
\end{figure}

经过不断完善和发展,形成了神经网络:
\begin{figure}[h]
	\centering
	\includegraphics[scale=0.6,angle=0]{images/4.png}
	\caption{神经网络}
	\label{4}
\end{figure}

神经网络使用剃度下降法求解最优值,剃度下降法是一类求解函数最小值的计算机算法。假设希望求解目标函数 \textbf{$f(x)=f(x_1,\dots ,x_n)$}的最小值,可以从一个初始点 \textbf{$x^{(0)}=(x_1^{(0)},\dots,x_n^{(0)}})$}开始,基于学习率 \textbf{$\alpha > 0$} 构建一个迭代过程:

$$x_1^{i+1}=x_1^{i}+\alpha \frac{\partial f}{\partial x_1}(x^{(i)})$$
$$ \dots $$
$$x_1^{i+1}=x_1^{i}+\alpha \frac{\partial f}{\partial x_1}(x^{(i)})$$

其中$x^{(i)}=(x_1^{(i),\dots,x_n^{(n)}}),i>=0$,一旦达到收敛条件的话,迭代就结束了。图\ref{5}是用剃度下降法求解$y=x^2$的最小值的过程,其中的点代表每次迭代后的值。

\begin{figure}[h]
	\centering
	\includegraphics[scale=0.7,angle=0]{images/5.png}
	\caption{剃度下降法图解}
	\label{5}
\end{figure}
\subsection{Bagging算法}
Bagging算法 (英语:Bootstrap aggregating,引导聚集算法),又称装袋算法,是机器学习领域的一种团体学习算法。最初由Leo Breiman于1996年提出。Bagging算法可与其他分类、回归算法结合,提高其准确率、稳定性的同时,通过降低结果的方差,避免过拟合的发生。
Bagging算法描述如下所示.
\begin{figure}[h]
	\centering
	\includegraphics[scale=0.7,angle=0]{images/6.png}
	\caption{Bagging 算法}
	\label{6}
\end{figure}

Bagging算法就是将多个分类或者回归器同时训练,最后取平均值作为最终结果。其原理如图\ref{7}所示,其中输入为自变量,即需要输入的训练数据,$Function 1-4$为4个不同的分类器或者回归器,$y_i$为它们的输出结果,Bagging模型最终求其平均作为最终结果。

\begin{figure}[h]
	\centering
	\includegraphics[scale=0.5,angle=0]{images/7.png}
	\caption{Bagging 算法图解}
	\label{7}
\end{figure}

\subsection{问题重述}
图 \ref{8} 是1898年到2012年明信片价格的变化,据此完成下列任务:
\begin{figure}[h]
	\centering
	\includegraphics[scale=0.6,angle=0]{images/8.png}
	\caption{明信片价格变化}
	\label{8}
\end{figure}

\noindent (一)
\begin{itemize}
\item 使用最小二乘法建立曲线回归模型,预测2020年明信片的价格.

\item 使用机器学习的方法求解高阶多项式模型并预测2020年明信片的价格.

\item 建立KNN回归模型并预测2020年明信片的价格.

\item 建立神经网络回归模型并预测2020年明信片的价格.

\item 在上述四种模型的基础上建立Bagging算法模型并预测2020年明信片的价格.
\end{itemize}

\noindent (二)
\begin{itemize}
\item 比较以上5种回归模型的结果并选出最合适的模型,给出最终结果.
\end{itemize}

\section{问题求解}
\subsection{数据预处理}
我们将数据输入为csv文件,并对年份进行了处理:
$$x_i=x_i-x_1 +1$$以便更好的求解模型。


\subsection{最小二乘法拟合$ax^n$}
假定需要拟合的曲线为$y=ax^2$,研究$f(x)=ax^2$的最小二乘估计,应用准则要求极小化$$S=\sum_{i=1}^m[y_i-f(x_i)]^2=\sum_{i=1}^m[y_i-ax_i^2]^2$$最优化的必要条件是导数 $dS/da$等于0:$$\frac{dS}{da}=-2\sum_{i=1}^mx_i^n[y_i-ax_i^2]=0$$从方程中解出a,得$$a=\frac{\sum x_i^2y_i}{\sum x_i^4}=0.001981$$
因此,回归方程为$$y=0.001981 * x^2$$,当$x=2020-1898+1=123$时,$$y=29.97$$


\begin{figure}[h]
	\centering
	\includegraphics[scale=0.5,angle=0]{images/9.png}
	\caption{拟合结果}
	\label{9}
\end{figure}

\subsection{剃度下降法拟合高阶多项式模型}
假设需要拟合的曲线为$$y=a_0+a_1x+a_2x^2+\dots+a_{10}x^{10}$$

损失函数为$$loss=\sum(y(x_i)-y_i)^2$$


初始化一组解$$a_0=a_1=\dots=a_{10}=1$$
求解
\begin{equation}
grad_{a_j}=\frac{dloss}{da_j}=2 \sum(y(x_i)-y_i) \sum x_i^j
\end{equation}
求解
\begin{equation}
a_i=a_i - \alpha *grad_{a_i}
\end{equation}
,其中 $\alpha = 0.001,i=1,2,\dots,10.$迭代2.1与2.2,直到$|grad_{a_i}|$<=0.001.通过计算机程序可求得:

$y=- 1.639136814e-15*x^{10} + 1.059936718e-12*x^9 - 0.0000000002960388319*x^8 $

$+ 0.00000004652752046*x^7 - 0.000004472787601*x^6 + 0.0002650337924*x^5$ 

$- 0.00900509148*x^4 + 0.126851489*x^3 + 1.240266286*x^2 $

$- 47.08797047*x + 46.72959737.$


当$x=123$时,$y=13.78$

\begin{figure}[h]
	\centering
	\includegraphics[scale=0.55,angle=0]{images/10.png}
	\caption{拟合结果}
	\label{10}
\end{figure}


\subsection{KNN回归}
我们使用KNN进行回归,得出以下结果:


\begin{figure}[H]
	\centering
	\includegraphics[scale=0.55,angle=0]{images/11.png}
	\caption{拟合结果}
	\label{11}
\end{figure}

当$x=123$时,$$y=29.67$$


\subsection{神经网络回归}
我们建立下面的神经网络,并进行训练.

\begin{figure}[H]
	\centering
	\includegraphics[scale=0.4,angle=0]{images/12.png}
	\caption{回归模型}
	\label{12}
\end{figure}

其中,学习速率 $\alpha=0.001$,最终拟合结果如图\ref{13}所示,当$x=123$时,$$y=34.00$$
\begin{figure}[h]
	\centering
	\includegraphics[scale=0.55,angle=0]{images/13.png}
	\caption{拟合结果}
	\label{13}
\end{figure}







\subsection{Bagging}
最后,我们使用Bagging算法对上面的四种结果取平均,其中,注意到,高阶多项式模型的误差比较大,因此Bagging算法选用其他三种回归算法。算法流程图如下。
\begin{figure}[h]
	\centering
	\includegraphics[scale=0.45,angle=0]{images/14.png}
	\caption{Bagging算法}
	\label{14}
\end{figure}

根据Bagging算法,可以求得$$y=\frac{1}{3}(34.00+29.67+29.97)=31.21$$


\section{模型比较}
\subsection{数据预处理}
为了能够更好的比较模型的准确度,我们从数据中抽取一半的数据用于建立模型,剩下的一半数据用于判断模型的准确度。数据分布如下:
\begin{figure}[h]
	\centering
	\includegraphics[scale=0.6,angle=0]{images/15.png}
	\caption{数据划分}
	\label{15}
\end{figure}


\subsection{最小二乘法回归}
假定需要拟合的曲线为$y=ax^2$,研究$f(x)=ax^2$的最小二乘估计,应用准则要求极小化$$S=\sum_{i=1}^m[y_i-f(x_i)]^2=\sum_{i=1}^m[y_i-ax_i^2]^2$$最优化的必要条件是导数 $dS/da$等于0:$$\frac{dS}{da}=-2\sum_{i=1}^mx_i^n[y_i-ax_i^2]=0$$从方程中解出a,得$$a=\frac{\sum x_i^2y_i}{\sum x_i^4}=0.001956$$
因此,回归方程为$$y=0.001981 * x^2$$.


\begin{figure}[h]
	\centering
	\includegraphics[scale=0.55,angle=0]{images/16.png}
	\caption{拟合结果}
	\label{16}
\end{figure}
测试数据的均方误差为
$$E=\frac{1}{m}\sum_{i=1}^m(f(x_i)-y_i)^2=4395.53.}$$


\subsection{剃度下降法拟合高阶多项式模型}
假设需要拟合的曲线为$$y=a_0+a_1x+a_2x^2+\dots+a_{10}x^{10}$$

损失函数为$$loss=\sum(y(x_i)-y_i)^2$$


初始化一组解$$a_0=a_1=\dots=a_{10}=1$$
求解
\begin{equation}
grad_{a_j}=\frac{dloss}{da_j}=2 \sum(y(x_i)-y_i) \sum x_i^j
\end{equation}
求解
\begin{equation}
a_i=a_i - \alpha *grad_{a_i}
\end{equation}
,其中 $\alpha = 0.001,i=1,2,\dots,10.$迭代2.1与2.2,直到$|grad_{a_i}|$<=0.001.通过计算机程序可求得:

$3.975590632e-13*x^10 - 0.0000000003290771782*x^9 $

$+ 0.0000001204184953*x^8 - 0.00002556825203*x^7 $

$+ 0.003471943435*x^6 - 0.3127761976*x^5 + 18.70628778*x^4 $

$- 717.4565232*x^3 + 16097.80278*x^2 - 165200.0207*x + 149802.2775$


\begin{figure}[h]
	\centering
	\includegraphics[scale=0.5,angle=0]{images/17.png}
	\caption{拟合结果}
	\label{17}
\end{figure}

测试数据的均方误差为
$$E=\frac{1}{m}\sum_{i=1}^m(f(x_i)-y_i)^2=177174.24.}$$



\subsection{KNN回归}
我们使用KNN进行回归,得出以下结果:


\begin{figure}[h]
	\centering
	\includegraphics[scale=0.5,angle=0]{images/18.png}
	\caption{拟合结果}
	\label{18}
\end{figure}

测试数据的均方误差为
$$E=\frac{1}{m}\sum_{i=1}^m(f(x_i)-y_i)^2=25636.22.}$$


\subsection{神经网络回归}
我们采用第二节的神经网络2.4模型进行训练.
其中,学习速率 $\alpha=0.001$,最终拟合结果如图\ref{13}所示,当$x=123$时,$$y=34.00$$
\begin{figure}[h]
	\centering
	\includegraphics[scale=0.5,angle=0]{images/19.png}
	\caption{拟合结果}
	\label{19}
\end{figure}



测试数据的均方误差为
$$E=\frac{1}{m}\sum_{i=1}^m(f(x_i)-y_i)^2=31.2013.}$$





\subsection{Bagging}
最后,我们使用Bagging算法对上面的四种结果取平均,其中,注意到,高阶多项式模型的误差比较大,因此Bagging算法选用其他三种回归算法。

\begin{figure}[h]
	\centering
	\includegraphics[scale=0.5,angle=0]{images/20.png}
	\caption{拟合结果}
	\label{20}
\end{figure}

测试数据的均方误差为:
$$E=\frac{1}{m}\sum_{i=1}^m(f(x_i)-y_i)^2=24614.14}$$

\subsection{最终结果}


\begin{table}[htbp]
	\centering
	\caption{不同回归方法的比较}
	
\begin{tabular}{|p{4.5cm}| p{3.5cm}| p{2.5cm}|}%会有5列,指定每列的居中形式,|表示每列中间有竖线分开
	\hline%每行之间由横线分开
	回归方法&2020年预测结果&均方差   \\%\\表示换行
	\hline
	最小二乘法曲线回归&29.97&2723090.92\\
	\hline
	高阶多项式回归&13.78&177174.24\\
	\hline
	KNN回归&29.67&25636.22\\
	\hline
	神经网络回归&34.00&31.2013\\
	\hline
	Bagging模型&31.21&24614.14\\
	\hline	
\end{tabular}
\end{table}	
\\

综上所述,选用神经网络回归效果最好,测试数据的均方误差为
$$E=31.2013.$$,2020年明信片价格预测结果为$$y=34.00$$


\section{总结}
从模型的结果来看,神经网络的拟合效果是最好的。其中KNN回归原理虽然简单,但是效果良好;而高阶多项式回归在数据集两侧的误差则特别大。Bagging模型的拟合效果虽然不如神经网络,但优于其他回归方法,具有提高准确率,降低过拟合的作用。





%\include{sections/Experimentalmaterials}
%\include{sections/Micro-organizationanalysis}
%\include{sections/Mechanicsperformanceanalysis}

%\include{sections/conclusion}
%\include{sections/thanks}

%\bibliographystyle{plain}
\bibliographystyle{unsrt}
\nocite{*}
\addcontentsline{toc}{section*}{参考文献}

\renewcommand\bibnumfmt[1]{\makebox[0.9cm][l]{[#1]}}
\setlength{\bibhang}{0em}


\begin{thebibliography}{99}
	\bibitem{book1}邓集贤,杨维权,司徒荣,邓永录.概率论与数理统计.下册(第四版).北京:高等教育出版社,2009.7
	
	\bibitem{art1}财务视角下新冠肺炎疫情对贵州工业发展影响的统计测度[C]//中国统计教育学会.2020年(第七届)全国大学生统计建模大赛优秀论文集.[出版者不详],2020:25.DOI:10.26914/c.cnkihy.2020.045581.
	
	\bibitem{art2}新冠肺炎疫情对青海省旅游业影响的统计研究——基于SARIMA模型[C]//中国统计教育学会.2020年(第七届)全国大学生统计建模大赛优秀论文集.[出版者不详],2020:16.DOI:10.26914/c.cnkihy.2020.045584.
	
	\bibitem{book2}周志华. 机器学习 [J]. 清华大学出版社, 2016, 8(28): 1–415.
	
	\bibitem{art3}[1]. 新冠疫情对海南旅游业影响的统计测度研究[C]//中国统计教育学会.2020年(第七届)全国大学生统计建模大赛优秀论文集.[出版者不详],2020:29.DOI:10.26914/c.cnkihy.2020.045597.
	
	\bibitem{book3}王燕.应用时间序列分析.中国大学出版社,2005.7
\end{thebibliography}	




%\include{sections/appendix}

\end{document}