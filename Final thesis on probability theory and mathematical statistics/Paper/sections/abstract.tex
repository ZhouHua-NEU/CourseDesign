\section*{\zihao{-3}\heiti 新冠疫情对秦皇岛旅游业影响的统计测度研究}
\section*{\zihao{3}\heiti 摘\ \ \ \ \ \ \ \ 要}

2020年爆发的新冠肺炎疫情对秦皇岛乃至全国的旅游业造成了巨大冲击。秦皇岛是著名的滨海旅游、休闲、度假胜地,旅游资源丰富。秦皇岛旅游业是秦皇岛的支柱产业之一,拉动了第三产业的发展。研究新冠肺炎疫情对秦皇岛旅游业的影响对未来秦皇岛旅游业及服务业发展、减小损失具有重要意义。

\textbf{第一步}:利用2012年1月至2019年12月的秦皇岛接待游客总量数据建立SARIMA模型,然后根据模型得出疫情爆发后 2020 年 1月至2021年12月接待游客总量的预测值,根据实际值与预测值的差距得到影响结果:\textbf{2020年秦皇岛旅游接待总人数减少80.87\%,2021年秦皇岛旅游接待总人数减少78.91\%,两年平均减少79.89\%。}

\textbf{第二步:}使用2012年1月-2019年12月的数据作为训练集训练LSTM神经网络,每24个月的数据用于预测下一个月的数据,然后使用2020,2021年的数据作为输入,即疫情发生后24个的月作为输入,逐步预测2022年的数据,最终与疫情发生前预测的2022年数据,以及实际的2020,2021年做对比,得出结论:\textbf{预计疫情对秦皇岛2022年旅游业的冲击依然很大,相比疫情前减少65.47\%;相比2020与2021年,疫情对秦皇岛的旅游业影响变小,秦皇岛旅游业相比2020,2021呈上升趋势,预计比2021年增长31.76\%。
}
\par
 
 
 \par
%空一行

\noindent \zihao{-4}{\heiti 关键词:}\textbf{新冠疫情};\textbf{接待游客};\textbf{SARIMA 模型};\textbf{LSTM神经网络};\textbf{影响因子}

\clearpage


%\section*{\zihao{-3}\bfseries Research on Microstructure and Mechanical Properties of Mg/Al Composite Sheet during Rolling Process}

%\hfill Author: xxxx

%\hfill Tutor: xxxx

%\section*{\zihao{3}\bfseries Abstract}

%As a green material in the 21st century, magnesium alloys have the advantages of low density, high specific strength, excellent thermal conductivity, good damping, shock absorption and impact resistance. However, the low yield strength, high notch sensitivity and poor corrosion resistance of magnesium alloys greatly limit the large-scale industrial application of them. Al/Mg/Al composite sheets have high specific strength and corrosion resistance of magnesium alloys with the advantages of good plasticity, toughness and corrosion resistance of aluminum and its alloys. It has broad application prospects in automotive, aerospace and other fields. In this paper, Al/Mg/Al composite sheets were prepared by composite rolling based on AZ31B magnesium alloy and 1060 pure aluminium sheets. The effects of process parameters on the microstructure and mechanical properties of the composite sheets were revealed.
%	\\
	%空一行
%	\\
%	\zihao{-4}{\bfseries Key words:\ }AZ31B magnesium alloy; Composite rolling; Microstructure; Mechanical property; Intermetallic compounds

%\clearpage




